\documentclass{beamer}
\usetheme{Pittsburgh}
\usecolortheme{rose}
\setbeamertemplate{navigation symbols}{}
\setbeamertemplate{blocks}[rounded]

\usepackage{xspace}
\usepackage{tikz-cd}

\title{Set-level mathematics}
\author{Joj Ahrens}
\date{UniMath School, Birmingham, UK, April 2019}

\newcommand{\fat}[1]{\textbf{#1}}
\newcommand{\constfont}[1]{\ensuremath{\mathsf{#1}}}
\newcommand{\C}{\ensuremath{\mathcal{C}}}
\newcommand{\Grp}{\constfont{Grp}}
\newcommand{\U}{\mathcal{U}}
\newcommand{\Nat}{\constfont{Nat}}
\newcommand{\Bool}{\constfont{Bool}}
\newcommand{\GrpStructure}{\constfont{GrpStructure}}
\newcommand{\hProp}{\constfont{hProp}}
\newcommand{\hSet}{\constfont{hSet}}
\newcommand{\isofhlevel}{\constfont{isofhlevel}}
\newcommand{\iscontr}{\constfont{iscontr}}
\newcommand{\isaprop}{\constfont{isaprop}}
\newcommand{\isaset}{\constfont{isaset}}
\newcommand{\eqv}[2]{\ensuremath{#1 \simeq #2}\xspace}
\newcommand{\carrier}{\constfont{carrier}}
\newcommand{\incl}{\constfont{incl}}
\newcommand{\isInjective}{\constfont{isInjective}}

%%% Transport (covariant) %%%
\newcommand{\transfib}[3]{\ensuremath{\mathsf{transport}^{#1}(#2,#3)\xspace}}
\newcommand{\Transfib}[3]{\ensuremath{\mathsf{transport}^{#1}\Big(#2,\, #3\Big)\xspace}}
\newcommand{\id}[3][]{\ensuremath{#2 =_{#1} #3}\xspace}

\newcommand{\outlinetitle}{Outline}

\AtBeginSection[]
{
  \begin{frame}<beamer>{\outlinetitle}
    \tableofcontents[currentsection]
  \end{frame}
}

\AtBeginSubsection[]
{
  \begin{frame}<beamer>{\outlinetitle}
    \tableofcontents[currentsection,currentsubsection]
  \end{frame}
}

\theoremstyle{definition}
\newtheorem{exercise}{Exercise}

\begin{document}
\begin{frame}
 \titlepage
\end{frame}

\begin{frame}
 \frametitle{\outlinetitle}
 \tableofcontents
\end{frame}

\section{Sets in UniMath}

\begin{frame}
 \frametitle{Definition of set}
 % Recall from Martin's talk
   \begin{align*}
    \iscontr(X) &:= \sum_{x:X}\prod_{y:X}x=y
    \\ \isaprop(X) &:= \prod_{x,y: X}x=y
    \\ \isaset(X) &:= \prod_{x,y: X}\isaprop(x=y)
   \end{align*}

   \begin{block}{}
      A \fat{set} is a type whose path types are all \emph{propositions}.
   \end{block}
\end{frame}

\begin{frame}
 \frametitle{Definition of h-Levels}
   \begin{align*}
       \isofhlevel(n,X) &: \Nat \to \U \to \hProp
    \\ \isofhlevel(0,X) &:= \iscontr(X)
    \\ \isofhlevel(S(n),X) &:= \prod_{x,x': X}\isofhlevel(n,x=x')
   \end{align*}

   \pause
   \begin{exercise}
     \[\left(\prod_{X:\U}\isaprop(X)\right)=\isofhlevel(1,X)\]
   \end{exercise}

   \pause
   \begin{block}{Hence:}
      A \fat{set} is a type of hlevel 2.
   \end{block}
\end{frame}

\begin{frame}
  \frametitle{Topological intuition}
  Recall:
  \begin{itemize}
  \item Each type $X$ is to be a thought of a \emph{space}
  \item For two points $a,b:X$, the type $a=_Xb$ is the \emph{space of paths} from $a$ to $b$
  \end{itemize}

  \begin{fact}[from topology]
    A (``nice'') space all of whose path spaces are contractible is (homotopy-)equivalent to a discrete space.
  \end{fact}

  \pause
  \begin{center}
  {\tiny This can be made more precise using the \emph{model in simplicial sets}.}
\end{center}
\end{frame}

\begin{frame}
 \frametitle{Closure properties}

   \begin{itemize}
    \item $\sum_{x:A}B(x)$ is a set if $A$ and all $B(x)$ are
    \item $A \times B$ is a set if $A$ and $B$ are
    \item $\prod_{x:A}B(x)$ is a set if all $B(x)$ are
    \item $A \to B$ is a set if $B$ is
   \end{itemize}

   \begin{itemize}
    \item Any property is a set
   \end{itemize}

   \begin{exercise}
          Do you know

     \begin{itemize}
      \item a type that is a set?
      \item a type for which you don't know (yet) whether it is a set?
     \end{itemize}

   \end{exercise}

\end{frame}

\begin{frame}
  \frametitle{What are sets good for?}

  \begin{itemize}
  \item Most ``traditional'' mathematics can be done in UniMath with sets (groups, rings, topological spaces, etc.) \pause
  \item Categories or the type of \emph{all} groups (rings, etc.) have h-level 3. \pause
  \item Higher category theory and synthetic homotopy theory require higher types.
  \end{itemize}
\end{frame}



\section{How to show that something is (not) a set?}

\begin{frame}
 \frametitle{Decidable equality}

    \begin{definition}
     A type $X$ is \fat{decidable} if we can write a term of type
     \[     X + \neg X \]
    \end{definition}

    \begin{definition}
     A type $X$ has \fat{decidable path-equality} if we can write a term of type
     \[  \prod_{x, x' : A} (x = x') + \neg(x=x') \]
     (that is, if all its paths types are decidable)
    \end{definition}

\end{frame}

\begin{frame}
  \frametitle{Hedberg's theorem}

  \begin{theorem}
   If a type $X$ has decidable equality, then it is a set.
  \end{theorem}

  In the problem session, we will show that $\Bool$ and $\Nat$ are sets.
  \pause
  \begin{block}{Note}
    Hedberg's theorem is \emph{hard}. There is also an easier proof that $\Bool$ and $\Nat$ are sets.
  \end{block}

\end{frame}

\begin{frame}
 \frametitle{Are all types sets?}
  \begin{block}{Is there a type that is not a set?}
    Great question! It depends:

    \begin{itemize}
     \item In ``spartan'' type theory some types can't be shown to be sets.
     \item Assuming univalence, some types can be shown not to be sets.
    \end{itemize}
  \end{block}
\end{frame}


\begin{frame}
 \frametitle{Another set}

  \begin{theorem}
    The type
     \[  \hProp_\U := \sum_{X : \U} \isaprop(X) \]
    is a set.
  \end{theorem}

  The proof relies on the univalence axiom for the unviverse $\U$.


  \begin{exercise}
    How would you generalize the above statement to any h-level?
    How would you attempt proving it?
  \end{exercise}

\end{frame}

\begin{frame}
 \frametitle{Types that are \fat{not} sets}

  \begin{exercise}
     Let $\U$ be a univalent universe that contains the type $\Bool$.
     Why is $\U$ not a set?
  \end{exercise}

  Which property of $\Bool$ does the proof of the above result exploit?

\end{frame}

\section{Subsets and quotients}

\begin{frame}
 \frametitle{Sets and propositions}

 Types representing properties of sets are usually propositions. \pause

 \begin{example}
   Given $f : X \to Y$,
   \[   \constfont{isInjective}(f) := \prod_{x, x' : X} f(x) = f(x') \to x = x' \]
   is not a proposition in general, but it is if $X$ and $Y$ are sets.
 \end{example}
\end{frame}

\begin{frame}
 \frametitle{Predicates on types}

    A \fat{subtype $A$} on a type $X$ is a map
    \[ A : X \to \hProp_U \]
  \begin{exercise}
   Show that the type of subtypes of any type $X$ is a set.
  \end{exercise}
  \pause
  The \fat{carrier} of a subtype $A$ is the type of elements satisfying $A$:
  \[ \carrier(A) := \sum_{x:X}A(x) \]
\end{frame}

\begin{frame}
 \frametitle{Predicates and injections}

 There is a canonical map $\incl_A:\carrier(A)\to{}X$\pause, namely
 \[pr1:\sum_{x:X}A(X)\to{}X\]

 \begin{exercise}
   \[\isaset(X)\to\isInjective(\incl_A)\]
 \end{exercise}
\end{frame}

\begin{frame}
 \frametitle{Predicates and injections}

 Conversely, given a map $f:A\to{}X$, we can form the function $\chi_f:X\to\U$ given by
 \[
   \chi_f(x):\equiv\sum_{a:A} f(a)=x
 \]\pause

 \begin{exercise}
   \[
     \isaset(X)\to\isInjective(f)\to\prod_{x:X}\isaprop(\chi_f(x))
   \]
 \end{exercise}\pause
 \begin{block}{Non-exercise}
   $\xi_f$ and $\incl_A$ establish an isomorphism between $X\to\hProp_\U$ and
   $\constfont{injections}(X):\equiv\sum_{A:\U}\isaset(A)\times\sum_{f:A\to{}X}\isInjective(f)$
 \end{block}
\end{frame}

\begin{frame}
 \frametitle{Relations on a type}

  A \fat{binary relation $R$} on a type $X$ is a map
  \[ R : X \to X \to \hProp_U \]\pause

  \begin{exercise}
   Show that the type of binary relations on $X$ is a set.
  \end{exercise}\pause

  Properties of such relations are defined as usual, e.g.,
  \[ \constfont{reflexive}(R) := \prod_{x : X} R(x)(x) \]\pause

  \begin{exercise}
   Formulate the properties of being symmetric, transitive, an equivalence relation.
  \end{exercise}

\end{frame}

\begin{frame}
  \frametitle{Set-level quotient}
  Given a set $X$ and relation $R$ on $X$, the \fat{quotient}
  \[ X \xrightarrow{p} X/R \]
  is defined by the property that any compatible map $f$ \fat{into a set $Y$} factors uniquely through $p$:
  \[
    \begin{tikzcd}[ampersand replacement=\&]
      X\ar[d, "p"']\ar[rd, "f"] \& \\
      X/R \ar[r, "\exists !~ f'"', dashed] \& Y
    \end{tikzcd}
  \]
  \begin{exercise}
    Formulate this condition precisely. % Since I couldn't fit it onto this slide.
  \end{exercise}
\end{frame}

\begin{frame}
  \frametitle{The quotient set}
  We can define the quotient $X/R$ of a set by an equivalence relation as the set of equivalence classes.\pause

  First we define for a subtype $A:X\to\hProp_U$
  \[
    \begin{split}
    \constfont{iseqclass}(A) :=
    &\left\lVert\carrier(A)\right\rVert\\
    \times&\prod_{x,y:A} R x y \to A x \to A y\\
    \times&\prod_{x,y:A} A x \to A y \to R x y
    \end{split}
  \]\pause

  Then we define
  \[
    X/R := \sum_{A:X\to\hProp_U}\constfont{iseqclass}(A)
  \]
\end{frame}

\section{Set-level mathematics}

\begin{frame}
 \frametitle{Paths between pairs}

  Given $B : A \to \U$ and $a,a' : A$ and $b : B(a)$ and $b' : B(a')$,
    \[ (a,b) = (a',b') \quad \simeq \quad \sum_{p : a = a'} \Transfib{B}{p}{b} = b' \]
  If $B(x)$ is a proposition for any $x:A$, then this simplifies to
  \[ (a,b) = (a',b') \quad \simeq \quad a = a' \]\pause

  \begin{exercise}
    Why?
  \end{exercise}
\end{frame}

\begin{frame}
 \frametitle{Example: natural numbers}

  An \fat{even natural number} is a pair consisting of a natural number and a proof of its evenness.
  \[
    \constfont{iseven}(n):\equiv\sum_{k:\Nat}k+k=n\quad\quad\quad
    \constfont{evennat}:\equiv\sum_{n:\Nat}\constfont{iseven}(n)
  \]\pause

  When comparing two even natural numbers, we want to compare them as numbers:
  \[  (n, p) = (n',p') \quad \simeq \quad n = n' \]\pause
  The type $\constfont{iseven}(n)$ hence should be a proposition.\pause
  \begin{exercise}
    It is!
  \end{exercise}

\end{frame}

\begin{frame}
 \frametitle{Groups}

    Traditionally (in set theory), a group is a quadruple $(G,m, e, i)$ of
      \begin{itemize}
       \item a set $G$
       \item a multiplication $m : G\times G \to G$
       \item a unit $e \in G$
       \item an inverse $i : G \to G$
      \end{itemize}
     subject to the usual axioms.

\end{frame}

\begin{frame}
 \frametitle{Groups in type theory}

    In type theory, a group is a (dependent) pair $(data, proof)$ where
      \begin{itemize}
       \item $data$ is a quadruple $(G,m,e,i)$ as above
       \item $p$ is a proof that these satisfy the usual axioms.
      \end{itemize}

      \pause
      We want to regard two groups $(data, proof)$ and $(data', proof')$ as being the same if $data$ is the same as $data'$.

      \pause
      This requires that the type encoding the group axioms be a \emph{proposition}.

      \pause
      This is in turn guaranteed as long as the underlying type $G$ is required to be a \emph{set}.

      \begin{exercise}
        Why?
      \end{exercise}
\end{frame}

\begin{frame}
  \frametitle{Group isomorphisms}

  The type of groups is
  \[ \Grp := \sum_{X : \hSet}\GrpStructure(X) \]

  \begin{block}{A \fat{group isomorphism $G \to G'$} is }
    \begin{itemize}
    \item a bijective function on the underlying sets $X \to X'$
    \item compatible with the group structures $S$ and $S'$ on $X$ and $X'$.
    \end{itemize}
  \end{block}

\end{frame}


\begin{frame}
 \frametitle{Identity is isomorphism for groups}
 \begin{align*}
   \id{G}{G'} \enspace
   & \simeq  \enspace \id{(X,S)}{(X',S')} \\
   & \simeq \enspace \sum_{p : \id{X}{X'}} \id{\transfib{\GrpStructure}{p}{S}}{S'} \\
   & \simeq \enspace \sum_{p : \id{X}{X'}} (\id{\transfib{Y \mapsto (Y \times Y \to Y)}{p}{m}}{m'})  \\
   &  \hspace{1.5cm} \times(\id{\transfib{Y \mapsto (Y \to Y)}{p}{i}}{i'}) \\
   & \hspace{1.5cm}  \times(\id{\transfib{Y \mapsto (1 \to Y)}{p}{e}}{e'}) \\
   & \simeq \enspace \sum_{f : \eqv{X}{X'}} \bigl(\id{f \circ m \circ (f^{-1} \times f^{-1})}{m'}\bigr)  \\
   &             \hspace{1.5cm}                            \times \bigl(\id{f \circ i \circ f^{-1}}{i'}\bigr) \\
   & \hspace{1.5cm}  \times(\id{f \circ e}{e'}) \\
   & \simeq \enspace  (G \cong G')
 \end{align*}



\end{frame}
\end{document}
