\documentclass[aspectratio=169]{beamer}

\newcommand{\Pause}{} % {\pause}

\usepackage{bbold,soul}
\usepackage{diagrams}
\newarrow{Into}{C}{-}{-}{-}{>}
\newarrow{Onto}{-}{-}{-}{-}{>>}
\newarrow{Eto}{.}{.}{.}{.}{>}

% \usefonttheme[onlymath]{serif}

\newcommand{\UA}{\operatorname{UA}}
\newcommand{\isPrime}{\operatorname{isPrime}}
\newcommand{\isSubsingleton}{\operatorname{isSubsingleton}}
\newcommand{\Prop}{\operatorname{Prop}}
\newcommand{\isProp}{\operatorname{isProp}}
\newcommand{\isSet}{\operatorname{isSet}}
\newcommand{\isSurjection}{\operatorname{isSurjection}}
\newcommand{\isSingleton}{\operatorname{isSingleton}}
\newcommand{\isContr}{\operatorname{isContr}}
\newcommand{\isEquivalence}{\operatorname{isEquivalence}}
\newcommand{\isIsomorphism}{\operatorname{isIsomorphism}}
\newcommand{\hasSection}{\operatorname{hasSection}}
\newcommand{\image}{\operatorname{image}}
\newcommand{\idtoeq}{\operatorname{idtoeq}}
\newcommand{\inr}{\operatorname{inr}}
\newcommand{\inl}{\operatorname{inl}}
\newcommand{\eqq}{\equiv}
\newcommand{\da}{\db{-}~}
\usepackage[all]{xy}
\usepackage{bbm,stmaryrd}

\newcommand{\wellinside}{\eqslantless}
\newcommand{\powerset}{\operatorname{\mathcal{P}}}
\newcommand{\U}{\operatorname{\mathcal{U}}}
\newcommand{\V}{\operatorname{\mathcal{V}}}
\newcommand{\K}{\operatorname{\mathcal{K}}}
\newcommand{\Opens}{\operatorname{\mathcal{O}}}
\newcommand{\Sierp}{\operatorname{\mathcal{S}}}
\newcommand{\tgs}[1]{{\scriptstyle \text{\dg{#1}}}}
\newcommand{\ts}[1]{{\scriptstyle \text{{#1}}}}

\renewcommand{\labelstyle}{\textstyle}

\definecolor{darkblue}{rgb}{0,0,0.5}
\newcommand{\db}{\textcolor{darkblue}}
\definecolor{darkgreen}{rgb}{0,0.5,0}
\newcommand{\dg}{\textcolor{darkgreen}}
\definecolor{grey}{rgb}{0.4,0.4,0.4}
\newcommand{\grey}{\textcolor{grey}}
\definecolor{darkred}{rgb}{0.5,0,0}
\newcommand{\dr}{\textcolor{darkred}}

\newcommand{\dual}[1]{{#1^{\wedge}}}
\newcommand{\doubledual}[1]{{#1^{\wedge\wedge}}}
\newcommand{\m}[1]{$\db{#1}$}
\newcommand{\mm}[1]{${#1}$}
\newcommand{\M}[1]{\[\db{#1}\]}
\newcommand{\MM}[1]{\[{#1}\]}

\newcommand{\N}{\mathbb{N}}
\newcommand{\R}{\mathbb{R}}
\newcommand{\NI}{\N_\infty}
\newcommand{\trunc}[1]{{\mathord{\parallel}#1\mathord{\parallel}}}
\newcommand{\trunci}[1]{{\mathord{\mid}#1\mathord{\mid}}}
\newcommand{\base}{\operatorname{base}}
\newcommand{\loo}{\operatorname{loop}}
\newcommand{\refl}{\operatorname{refl}}
\newcommand{\idp}{\operatorname{idp}}
\newcommand{\Id}{\operatorname{Id}}
\newcommand{\Path}{\operatorname{Path}}
\newcommand{\id}{\operatorname{id}}
\newcommand{\transport}{\operatorname{transport}}
\newcommand{\one}{1} % {\mathbbm{1}}
\newcommand{\Z}{\mathbbm{Z}}
\newcommand{\ap}{\operatorname{ap}}
\newcommand{\constant}{\operatorname{constant}}
\newcommand{\eqdef}{\stackrel{{\rm def}}{=}}

\title{{\Large Univalent Foundations}}

\author{Mart\'in H\"otzel Escard\'o}
\institute{University of Birmingham, UK}
\date{\small \dg{UniMath School, Birmingham, UK, December 2017}}

\begin{document}

\begin{frame}
  \titlepage
\end{frame}

\begin{frame}
  \frametitle{Rosetta stone}

  \url{https://github.com/UniMath/UniMath/blob/master/USAGE.md}

  Scroll down to see a notation table.

\end{frame}

\begin{frame}
  \frametitle{\dg{Exercise}s and self-study}

  \begin{itemize}
  \vfill \item
  There are lots of hidden exercises disguised as claims in these notes.

  \vfill
  \item There is also important material which we won't have time to cover in the lecture.
  \end{itemize}

\end{frame}

\begin{frame}
  \frametitle{Univalent mathematics in outline}

  \begin{enumerate}
  \vfill \item Developed in a \db{univalent type theory}.
  \vfill \item Crucially relies on \db{identity types} or \db{type of paths between two points}.
  \vfill \item Certain types behave like sets \grey{(e.g.\ natural numbers).}
  \vfill \item But there are more general types \grey{(e.g.\ the type of groups, the type of categories)}.
  \vfill \item Certain types behave like propositions \grey{(e.g.\ the empty type and the unit type.)}
    \begin{itemize}
    \vfill \item Logical connectives and quantifiers become constructions of types.
  \vfill \item Proofs become constructions of elements of types.
  \vfill \item \db{Univalent logic} is constructive by default,
        \\ but compatible with non-constructive principles \grey{(excluded middle, choice)}.
        \end{itemize}
  \end{enumerate}

\end{frame}


\begin{frame}
  \frametitle{What a univalent type theory is \framebox{for}}

  \begin{enumerate}
  \vfill \item \db{General mathematics} performed in a certain way \grey{(univalent mathematics)}.

  \vfill \item \db{Synthetic homotopy theory} \grey{(homotopy type theory, or HoTT)}.
  \end{enumerate}

\vfill

This lecture is about \emph{\db{general mathematics}} in univalent type theory: \\[1ex]
\grey{\framebox{(1)} rather than (2).}

\end{frame}

\begin{frame}
  \frametitle{What a univalent type theory \framebox{is}}

\vfill

  \db{As a first approximation, it is a \framebox{\dg{mathematical language}} for expressing definitions, theorems and proofs that is
  \framebox{\dg{invariant under isomorphism:}}}

\vfill
\begin{quote} \em
If you can say something about the object \m{X}, and \m{Y} is an object \db{isomorphic} to \m{X}, then you can say the same thing about \m{Y}: \\[3ex]
\begin{quote}
\m{P(X)} and \m{X \simeq Y} together imply \m{P(Y)}.
\end{quote}
\end{quote}

\end{frame}


\begin{frame}
  \frametitle{A precursor}

\begin{enumerate}
\vfill \item It may also happen for a language for mathematics that it makes impossible  to distinguish isomorphic objects, \db{even if it wasn't designed with that purpose in mind.}

\vfill \item Such a language is  \db{intensional Martin-L\"of type theory} (MLTT).
\vfill \item For example, in MLTT we can't exhibit any property \m{P} of \m{\N} that \m{\N \times \N} hasn't. \grey{(Metatheorem.)}
\vfill \item In \emph{set theory}, they can be distinguished by the property \m{P(X) \eqq (0 \in X)}. \\[1ex]

\grey{(There isn't a global membership relation in MLTT.)} \\[1ex]

\end{enumerate}

\end{frame}

\begin{frame}
  \frametitle{Some univalent type theories}

  \begin{enumerate}
  \vfill \item A spartan MLTT \db{+} univalence axiom \db{+} resizing rules. \\[1ex]
  \grey{(UniMath)}

  \vfill \item A more generous MLTT \db{+} univalence axiom \db{+} higher-inductive types. \\[1ex]
  \grey{(Homotopy type theory book)}

  \vfill \item Cubical Type Theory.\\[1ex]
  \begin{itemize}
  \item  Has univalence as a theorem.
  \item  Has some higher-inductive types.
  \item  Extends a spartan MLTT.
  \item  Has intrinsic computational content.
  \end{itemize}
  \end{enumerate}

  I will formulate the univalence axiom after we have developed enough
  material.

\end{frame}

\begin{frame}
  \frametitle{A spartan \grey{(intensional)} Martin-L\"of type theory}

\vfill

  \begin{enumerate}
  \vfill \item Base types \m{\mathbb{0}}, \m{\mathbb{1}},  \m{\mathbb{2}}, \m{\N}.
  \vfill \item Cartesian products \m{A \times B} and \m{\Pi(x:X), A(x)}.
  \vfill \item Function types \m{X \to A} also written \m{A^X}.
  \vfill \item Sums \m{A + B} and \m{\Sigma(x:X), A(x)}.
  \vfill \item A large type \m{\U} of small types (universe). \\[1ex] Sometimes an extra large type \m{\V} of large types with \m{\U : \V}.
  \vfill \item An identity type \m{\Id_X(x,y)} for any \m{X} in \m{\U} or \m{\V} and \m{x,y:X}.
\end{enumerate}

\vfill

\end{frame}

% \begin{frame}
%   \frametitle{A spartan (intensional) Martin-L\"of type theory}

% \vfill

%   \begin{enumerate}
%   \vfill \item Base types \m{\mathbb{0}}, \m{\mathbb{1}},  \m{\mathbb{2}}, \m{\N}.
%   \vfill \item Cartesian products \m{A \times B} and \m{\Pi(x:X), A(x)}.\\[1ex]
%   \begin{itemize}
% \item \grey{Elements of \m{A \times B} are pairs \m{(a,b)} with \m{a:A} and \m{b:B}.} \\[1ex]

% \item \grey{Elements of \m{\Pi(x:X), A(x)} are functions that map \m{x:X} to some \m{a : A(x)}}.
%   \end{itemize}
%   \vfill \item Function types \m{X \to A} also written \m{A^X}.
%   \vfill \item Sums \m{A + B} and \m{\Sigma(x:X), A(x)}.\\[1ex]
%   \begin{itemize}
%   \item An element of \m{A+B} is either an element of \m{A} or an element of \m{B} with a tag. \\[1ex]
%   \item
% Elements of \m{\Sigma(x:X), A(x)} are pairs \m{(x,a)} with \m{x:X} and \m{a:A(x)}.
%   \end{itemize}
%   \vfill \item A large type \m{\U} of small types (universe). \\[1ex] Sometimes a extralarge type \m{\V} of large types. \\[1ex]
%   \vfill \item An identity type \m{\Id_X(x,y)} for any \m{X : \U} and \m{x,y:X}. \\[1ex]
%   \begin{itemize}
%   \item This type collects the ways in which the points \m{x} and \m{y} are identified.
%   \end{itemize}
% \end{enumerate}

% \vfill

% \end{frame}

\begin{frame}
  \frametitle{Identity type discussion}

  \begin{enumerate}
\vfill \item The type \m{\Id_X(x,y)} collects the ways in which the points \m{x,y:X} are identified.
  \vfill \item For some types the identity type can be defined from the other types.
  \vfill \item E.g.\ for the type of natural numbers, it can be constructed by double induction:
  \db{\begin{eqnarray*}
    \Id_\N(0,0) & \eqq & \mathbb{1}, \\
    \Id_\N(0,n+1) & \eqq & \mathbb{O}, \\
    \Id_\N(m+1,0) & \eqq & \mathbb{O}, \\
    \Id_\N(m+1,n+1) & \eqq & \Id_\N(m,n).
  \end{eqnarray*}}

  \vspace*{-4ex}

  \vfill \item In this example, \m{\Id_X(x,y)} is always a subsingleton.
  \vfill \item In general, there may be more identifications of \m{x} and \m{y} in the identity type.
  \vfill \item We often write \m{x =_X y} or simply \m{x = y} to denote \m{\Id_X(x,y)}.
  \vfill \item We write \m{\eqq} for definitions.
  \end{enumerate}

\end{frame}


\begin{frame}
  \frametitle{Universe discussion}

  \begin{itemize}
  \vfill \item We assume a large type \m{\U} whose elements are small types.
\vfill \item This has many uses, e.g.:
  \end{itemize}
\begin{enumerate}
\vfill \item Define type families as functions \m{X \to \U}, for example by induction if \m{X} is the type of natural numbers. \\[1ex]
\grey{(Like in the previous slide.)}
\vfill \item Define properties of elements of types, e.g. \m{\operatorname{isEven} : \N \to \U}.
\vfill \item Define the type of groups:
\M{Group \eqq \Sigma(G:\U), \isSet(G) \times \Sigma(\cdot : G \times G \to G), \Sigma(e:G), (\Pi(x:G), x \cdot e = x) \times \cdots} \\[1ex]
Define the type of categories. Or of topological spaces.
Etc.
\end{enumerate}

\end{frame}


\begin{frame}
  \frametitle{Two kinds of type-theoretic logic}

  \begin{itemize}
  \vfill \item Curry--Howard logic.

  \begin{enumerate}
  \item Propositions are types.
  \item Proofs are elements of types.
  \end{enumerate}

  \vfill \item Univalent logic.
  \begin{enumerate}
  \item Propositions are \db{subsingleton} types.
  \grey{(As in topos logic.)}

  \item Proofs are again elements of types.
  \end{enumerate}
  \end{itemize}

\vfill \da Both are constructive by default.

\vfill \da But we can consistently postulate excluded middle and choice if we wish. \\
\grey{~~(At the expense of losing implicit computational content.)}

\end{frame}

\begin{frame}
  \frametitle{Curry--Howard propositional logic}

Given two propositions \grey{(that is, types)} \m{A} and \m{B}, we define
\begin{enumerate}
\vfill \item \dg{Conjunction.} \mm{\dr{A \wedge B} \eqq \db{A \times B}}. \\[1ex]
  \begin{quote}
``A proof of \m{A\wedge B} is a pair \m{(a,b)} consisting of a proof \m{a} of \m{A} and a proof \m{b} of \m{B}.''
  \end{quote}
\vfill \item \dg{Disjunction.} \mm{\dr{A \vee B} \eqq \db{A + B}}. \\[1ex]
  \begin{quote}
``A proof of \m{A\vee B} is either a proof of \m{A} or a proof of \m{B}.''
  \end{quote}
\vfill \item \dg{Implication.} \mm{\dr{A \to B}} is the function space, which has the same notation. \\[1ex]
  \begin{quote}
``A proof of \m{A \to B} transforms any proof of \m{A} into a proof of \m{B}.''
  \end{quote}

\vfill \item \dg{Negation} \mm{\dr{\neg A} \eqq \db{A \to \mathbb{0}}}. \\[1ex]
\end{enumerate}
\end{frame}

\begin{frame}
  \frametitle{Curry--Howard quantifiers}

Given a family \m{A} of propositions (that is, types) indexed by a type \m{X}, we have:
\begin{enumerate}
\vfill \item \dg{Universal quantification:} \quad \mm{\dr{\forall(x:X), A(x)} \eqq \db{\Pi(x:X), A(x)}}.

\vfill

\begin{quote}
  ``A proof of \m{\forall(x:X), A(x)} is a function that gives a proof of \m{A(x)} for any given \m{x:X}.''
\end{quote}


\vfill \item \dg{Existential quantification:} \quad \mm{\dr{\exists(x:X), A(x)} \eqq \db{\Sigma(x:X), A(x)}}.

\vfill

\begin{quote}
  ``A proof of \m{\exists(x:X), A(x)} is a pair \m{(x,a)}  consisting of a witness \m{x:X} and a proof \m{a} of \m{A(x)}.''
\end{quote}

\end{enumerate}

\end{frame}

\begin{frame}
  \frametitle{Curry--Howard logic}

  \begin{tabular}{|| l | l | l | l ||}
    \hline \hline
    false &  \m{\bot}         & \m{\mathbb{0}}  & empty type \\ \hline
    true &  \m{\top}        & \  & any type for which a point can be exhibited \\ \hline
    and &  \m{\wedge} & \m{\times} & cartesian product of two types \\ \hline
    or &  \m{\vee} & \m{+} & disjoint sum of two types \\ \hline
    implies  & \m{\to} & \m{\to} & function space \\ \hline
    for all  & \m{\forall} & \m{\Pi} & product of a type family \\ \hline
    for some &  \m{\exists} & \m{\Sigma} & disjoint sum of a type family \\ \hline
equals & \m{=} & \m{\Id} & identity type \\ \hline
    \hline
  \end{tabular}

\vfill

Martin-L\"of introduced the identity type precisely to extend
Curry-Howard logic with equality.

\end{frame}

\begin{frame}
  \frametitle{\dg{Example:} there are infinitely many prime numbers}

\M{f : \Pi(n:\N), \Sigma(p:\N), (p > n) \times \isPrime(p).}


\begin{enumerate}
\vfill \item
A point of this type is a function \m{f} that for any \m{n:\N} gives a prime~\m{p > n}.

\vfill \item The type \m{n > p} can be defined by induction on \m{n} and \m{p} like we defined the identity type \m{m = n}. \\[1ex]

\grey{Or in many other equivalent ways, e.g. \m{(n > p) \eqq \Sigma(k : \N), p + k + 1 = n}, after we have defined addition by induction.}

\vfill \item After we define multiplication by induction, we define \m{\operatorname{isPrime}(p)} in the usual way.

\end{enumerate}
\vfill
\dr{A lot of mathematics can be formulated and proved in this way.}

\end{frame}

\begin{frame}
  \frametitle{\dg{Exercise}}

  \begin{enumerate}
  \item Define \m{>} of type \m{\N \to \N \to \U} by double induction on \m{\N}.
  \item Prove that \m{n > p} implies \m{\Sigma(k : \N), p + k + 1 = n} and conversely.
  \item Define \m{\operatorname{isPrime}}.
  \end{enumerate}

\end{frame}

\begin{frame}
  \frametitle{Definition of the identity type (or path type)}

  \begin{itemize}
    \vfill \item
    Given a type \m{X : \U} and any two points \m{x,y:X}, we have a type \M{\Id_X \, x \, y : \U,} whose
    elements are called \db{paths} from \m{x} to \m{y}.
    \vfill
    We also use the following two synonyms \grey{(often with the subscript \m{X} elided)}:
    \M{\Path_X \,x \,y, \qquad \qquad x =_X y.}
    \vfill \item
    This comes with a specified path \m{\idp x : \Id \, x \, x}, called \db{identity path}, for every point \m{x:X}.
    \vfill \item
    And with an induction principle.
  \end{itemize}

\end{frame}

\begin{frame}
  \frametitle{Definition of the identity type (or path type)}

  Notice that \m{\Id : X \to X \to \U} is a type family and \m{\idp : \Pi(x:X), \Id \, x \, x}.

  \begin{itemize}
    \vfill \item \db{Recursion principle}. Given any other type family \m{Id' : X \to X \to \U} together with a function \m{\idp' : \Pi(x:X), \Id' \, x \, x}, we have a function
    \M{h : \Pi(x,y:X), \Id\,x\,y \to Id'\,x \, y  }
    with
    \M{h\,x\,x\,(\idp x) \eqq idp' \, x.}
   \end{itemize}

    This \db{doesn't} say that \m{\idp \, x} is the only path.

\vfill

    What this says is that \m{h} is uniquely determined by its effect on identity paths.

\vfill

    \grey{(Compare to the Yoneda Lemma, if you know it, which says that a
    natural transformation of a certain kind is uniquely determined by
    its effect on identity morphisms.)}

 \end{frame}


\begin{frame}
  \frametitle{Definition of the identity type (or path type)}

  The recursion principle allows to define many useful things, including:
  \begin{enumerate}
  \item Path inverse: \m{\Pi(x,y:X), x = y \to y = x}.
  \item Path composition: \m{\Pi(x,y,z:X), x = y \to y = z \to x = z}.
  \item Map-on-path: \m{\Pi(f:X \to Y),\Pi(x,y:X), x = y \to f \, x = f \, y}.
  \item Transport: \m{\Pi(A : X \to \U), \Pi (x,y:X), x = y \to A \, x \to A \, y}.
\end{enumerate}

But the recursion principle is not enough to e.g.\ show that
\begin{enumerate}
\item Path inversion is an involution.
\item Path composition is associative with identity paths as a left- and right-neutral elements.
\item Map-on-path is functorial with respect to path identity and composition.
\item A transport along any path is invertible.
\end{enumerate}


 \end{frame}

\begin{frame} \frametitle{\dg{Exercise}}

  Define path inversion, composition, map-on-path, and transport.

\end{frame}


 \begin{frame}
  \frametitle{Definition of the identity type (or path type)}

  \begin{itemize}
  \item \db{Induction principle}. If \m{P : \Pi(x,y), x=y \to \U} is any type family and \m{b : \Pi(x:X), P\,x\,x\,(\idp x)} is any function, then we have a function
    \M{h : \Pi(x,y:X), \Pi(p:x=y), P\,x\,y\, p}
    with
    \M{h \, x \, x\, (\idp x) = b\, x.}
  \end{itemize}

  \vfill

  This says that to ``prove'' that \m{P} ``holds'' for all paths \m{p}
  between any two points \m{x} and~\m{y}, it suffices to prove the
  ``base case'' \m{b} that it holds for all identity paths.

  \vfill

  The recursion principle is the particular case with \m{P \, x\,y\,p
    \eqq \Id' \,x \,y} that depends on the end-points of \m{p}, but not
  on \m{p} itself.

 \end{frame}

\begin{frame} \frametitle{\dg{Exercise}}

  Formulate and prove the above properties of inversion, composition, map-on-path, and transport.

\end{frame}


 \begin{frame}
  \frametitle{Definition of the identity type (or path type)}

  Summary:
  \begin{itemize}
  \vfill \item Type former \m{\Id} \grey{(also written \m{=} or \m{\Path}).}
  \vfill \item Identity paths \m{\idp} \grey{(also written \m{\refl} in the original literature).}
  \vfill \item Induction principle \grey{(also called \m{J}).} \\[1ex] \grey{To prove something for all paths between any two points, it is enough to prove it for all identity paths of all points.}
  \end{itemize}

 \end{frame}

 \begin{frame}
  \frametitle{\dg{Preparation:} Univalence gives function extensionality (funext)}

Pointwise equal functions are equal:
\[\dg{\Pi(X:\U), \Pi(A:X \to \U), \Pi(f,g:\Pi(x:X), A(x)),} \dr{(\Pi(x:X), f(x)=g(x))} \to \db{f=g}.\]

\begin{itemize}
\vfill \item
Something we don't have in intensional Martin-L\"of type theories.
\vfill \item
We don't have univalence yet, so we assume \db{funext} when needed, for the moment.
\end{itemize}

\end{frame}

\begin{frame}
  \frametitle{Univalent propositions}

Or \emph{propositions}, for short (also known as \emph{h-propositions} or \emph{mere propositions}).

\vfill

  \begin{enumerate}
  \vfill \item A type \m{X} is a \db{proposition} if any two of its elements are equal:
\MM{\dr{\isProp(X)} \eqq \db{\Pi(x,y:X), x=y}.}

\vfill

We also say that \m{X} is a \db{subsingleton}.
\MM{\dr{\Prop} \eqq \db{\Sigma(P:\U), \isProp(P)}.}

\vfill

  \vfill \item We have \m{\text{funext} \to \isProp(\isProp(X))}.

\vfill

  \grey{Being a proposition is itself a proposition.}

\vfill

\vfill \item Assuming \db{funext}, also any two maps into the same proposition are equal.

\end{enumerate}

\vfill

\end{frame}

\begin{frame} \frametitle{\dg{Exercise}}

  \begin{enumerate}
  \item  \m{\text{funext} \to \isProp(\isProp(X))}.
  \item Any two functions into a proposition are equal, assuming funext.
  \end{enumerate}

\end{frame}



\begin{frame}
  \frametitle{Example of a non-trivial type that is a univalent proposition}

\M{\Sigma(n:\N), \isPrime(n) \times \text{(n is the difference of two squares of primes).}}

\begin{enumerate}
\vfill \item Although propositions are subsingletons, they are not necessarily \db{``proof-irrelevant''}.
\begin{itemize}
\vfill \item They have \db{information content}.
\vfill \item The number \m{5} can be extracted from any proof of this proposition.
\end{itemize}

\vfill \item But for the above \db{type} to really be a \db{proposition}, we additionally need:
  \begin{itemize}
\vfill \item \db{propositional extensionality} \grey{(any two logically equivalent
propositions are equal)}.
\vfill \item Which is not provable in MLTT, but follows from univalence, like \db{funext}.
\end{itemize}

\end{enumerate}

\vfill
\end{frame}

\begin{frame}
  \frametitle{A subtler example \grey{(the mystery of MLTT's identity type)}}

Although the type \m{x=y} need not be a subsingleton for \m{x,y:X}, the type
\M{\Sigma(x:X), x = y}
is \db{always} a subsingleton for any \m{y:Y}, and in fact even a \db{singleton}, or \db{contractible}:
\[
\dr{\isSingleton(A)} \eqq \db{\Sigma(a_0:A), \Pi(a:A), a = a_0}.
\]
\grey{This doesn't require propositional or function extensionality, or anything beyond MLTT.}

\vfill

\dg{Theorem of MLTT.} \m{\Pi(y:X), \isSingleton(\Sigma(x:X), x = y).}

\vfill

\dg{Non-Theorem of MLTT.} \m{\text{\st{$\Pi(x,y:X), \isSubsingleton(x = y)$}.}}

\end{frame}

\begin{frame}\frametitle{\dg{Exercise}}

Show that \m{\Sigma(x:X), x = y} is contractible by path induction.

\end{frame}

\begin{frame}
  \frametitle{An even subtler, crucial example (Voevodsky)}

Let \m{f:X \to Y} be a function of two types \m{X} and \m{Y}.
  \begin{enumerate}
  \vfill \item The type \[\dr{\isIsomorphism(f)} \eqq \db{\Sigma(g:Y \to X),  (g \circ f = \id_X) \times (f \circ g = \id_Y)}\] need not be a proposition.

\vfill

\grey{Because of the potential presence, in MLTT, of higher-dimensional types.}

\vfill \item However, the type
\[
\dr{\isEquivalence(f)} \eqq \db{\Pi(y:Y), \isSingleton(\Sigma(x:X), f(x)=y)}
\]
always is a proposition, assuming functional extensionality.

\vfill \grey{(And the latter is a \db{retract} of the former.)}
  \end{enumerate}

\end{frame}

\begin{frame}
  \frametitle{\dg{Exercise}}

Let \m{X} and \m{Y} be types and \m{f:X \to Y}.
 \begin{enumerate}
  \item Using funext, show that \m{\isProp(\isEquivalence(f))} \grey{(not hard)}.
  \item Define a map \m{s : \isEquivalence(f) \to \isIsomorphism(f)} \grey{(not hard)}.
  \item Define a map \m{r : \isIsomorphism(f) \to \isEquivalence(f)} \grey{(rather hard)}.
   \item Conclude that \m{r(s(\phi)) = \phi} for any \m{\phi :
   \isEquivalence(f)} using the fact that \m{\isProp(\isEquivalence(f))} \grey{(direct)}.
  \end{enumerate}

\end{frame}

\begin{frame}
  \frametitle{\dg{Exercise}}

  Define \m{\idtoeq}. You will need to prove that the identity functioni is an equivalence.

\end{frame}


\begin{frame}
  \frametitle{Formulation of the univalence axiom, and consequences}

\begin{enumerate}
\vfill \item $\dr{X \simeq Y} \eqq \db{\Sigma(f:X \to Y), \isEquivalence(f)}.$
\vfill \item By path recursion, we have a function \m{\idtoeq : \Pi(X,Y:\U), X = Y \to X \simeq Y} that maps identity paths to identity equivalences.
\vfill \item \m{\UA \eqq \Pi(X,Y:\U), \isEquivalence(\idtoeq \, X \, Y)}.
\end{enumerate}

\begin{itemize}
\vfill \item
\dg{Theorem of MLTT (Voevodsky).} The type \m{\UA} is a proposition.

\vfill \item
\dg{Metatheorem (Voevodsky).} \m{\UA} is consistent with MLTT. \grey{(Simplicial set model.)}

\vfill \item
\dg{Theorem of MLTT+UA.} \m{P(X)} and \m{X \simeq Y} imply \m{P(Y)} for any \m{P : \U \to \U}.

\vfill \item
\dr{Theorem of spartan MLTT with two universes.} The univalence axiom formulated with \db{crude isomorphism} rather than \db{equivalence} is \framebox{\dr{false}}.
\end{itemize}

\end{frame}

\begin{frame}
  \frametitle{Some consequences of univalence}

  \begin{itemize}
  \vfill \item Functional and propositional extensionality. \\[1ex]
  \grey{Which we already needed.} \\[1ex]
  \db{Univalence is a generalized extensionality axiom for intensional MLTT.}
  \vfill \item In the type of groups, the paths are in one-to-one correspondence with group isomorphisms. \\[1ex]
  Therefore isomorphic groups have the same properties (by transport).
  \end{itemize}

\end{frame}

 \begin{frame}
   \frametitle{Voevodsky's stratification of types by \db{hlevels}}

   \begin{enumerate}
   \vfill \item[0.] Contractible types.
   \vfill \item[1.] Types whose path spaces are all contractible. \\[1ex]

   \dg{Exercise}: A type is of hlevel 1 iff it is a proposition in the sense defined above.

     \vfill \item[2.] Types whose path spaces are all propositions. \\[1ex]

     These are known as \db{sets}. Between any two points, there is at most one path.

     \vfill \item[3.] Types whose path spaces are all sets. \\[1ex]

     These are known as \db{1-groupoids}. An example is the type of groups. \\[1ex]

     Remember that we defined a group to be a \emph{set} together with a binary operation etc., rather than an arbitrary \db{type}.

$\vdots$

       \vfill \item[n+1.] Types whose path spaces are all of hlevel \m{n}.


   \end{enumerate}



 \end{frame}

 \begin{frame}
   \frametitle{Voevodsky's stratification of types by \db{hlevels}}

   \begin{itemize}
   \vfill \item If a type is of hlevel \m{n}, then it is of hlevel \m{m} for any \m{m>n}. \\[1ex]

     \grey{(Contractible types are propositions, propositions are sets, sets are (trivial) 1-groupoids, etc.)}


     \vfill \item There is no reason why all types must have a finite hlevel. \\[1ex] Paths between paths between paths between paths \dots can go on forever in a non-trivial way.

   \vfill \item In the absence of univalence, it is consistent that all types are sets.
   \end{itemize}

 \end{frame}

\begin{frame}
  \frametitle{Another example of when Curry--Howard goes wrong: \framebox{image}}

Define the image of a function \m{f:X \to Y} in the usual way, translated to Curry-Howard:
\M{\dr{\image f} \eqq \db{\Sigma(y:Y), \Sigma(x:X), f(x)=y}.}

\begin{itemize}
\vfill \item
This is the type of points \m{y:Y} for which we have some \m{x:X} with \m{f(x)=y}.
\vfill \item \dr{Trouble:} \m{\image f \simeq X.} \\[1ex]
\framebox{This is not what we expect.}

% \vfill

% \grey{(\dg{Exercise}: Also in set theory, the graph of \m{f:X\to Y} is isomorphic to \m{X}.)}

\vfill \item \dr{Example.} We don't expect the image of the unique function \m{\mathbb{2} \to \mathbb{1}} to be isomorphic to \m{\mathbb{2}}. \\[1ex]

\grey{We expect the image to be a subtype of \m{\mathbb{1}}.}

%\vfill \item \dr{Can we fix this} in the same way we fixed isomorphism with equivalence?

\end{itemize}

\vfill

\grey{Univalent logic fixes such things in the same way as topos logic.}

\end{frame}

\begin{frame}
  \frametitle{\dg{Exercise}}

  Formulate and prove the troublesome claim.

\end{frame}


\begin{frame}
  \frametitle{Propositional truncation (or reflection)}

  \begin{enumerate}
  \vfill \item A \db{propositional truncation} of a type \m{X}, if it exists, is the universal solution to the problem of mapping \m{X} to a proposition:
  \db{\begin{diagram}
\text{\dg{arbitrary type}} \qquad &     X & \rTo & \trunc{X} & \quad \text{\dg{proposition}} \\
&       & \rdTo  & \dEto & \\
&       &            & P & \quad \text{\dg{proposition.}}
  \end{diagram}}

\m{\trunc{X}} is required to be the smallest proposition \m{X} maps into.

\vfill \item Several kinds of types can be shown to have truncations
in MLTT.

\vfill \item There are a number of ways to extend MLTT to get truncations for \emph{\db{all}} types.

\grey{(Such as resizing + funext, or higher inductive types.)}

  \end{enumerate}

\vfill

\end{frame}

\begin{frame}
  \frametitle{Example}

  \vfill

  \db{\begin{diagram}
\text{\dg{Not always a proposition}} \qquad & \isIsomorphism f & \rTo & \isEquivalence f & \quad \text{\dg{always proposition}} \\
&       & \rdTo  & \dEto & \\
&       &            & P & \quad \text{\dg{proposition.}}
  \end{diagram}}

\vfill

\end{frame}

\begin{frame}
  \frametitle{Example}

  \vfill

  Let \m{f: \N \to \N} and define \M{\dr{\Sigma^{\min}(n:\N), f\,n = 0} \eqq  \db{\Sigma(n:\N), (f \, n = 0) \times \Pi(m:\N), f\,m=0 \to n \le m}.}

  \vfill

   \begin{diagram}%[small]
 \text{\dg{Not always a proposition}}  \qquad & \Sigma(n:\N), f\,n = 0 & \rTo & \Sigma^{\min}(n:\N), f\,n = 0 & \quad \text{\dg{a proposition}} \\
 &       & \rdTo  & \dEto & \\
 &       &            & P & \quad \text{\dg{proposition.}}
   \end{diagram}

\vfill

\end{frame}

\begin{frame}
  \frametitle{\dg{Exercise}}

  Prove that \m{\Sigma^{\min}(n:\N), f\,n = 0} is the propositional
  truncation of \m{\Sigma(n:\N), (f \, n = 0)} by showing it has the
  right universal property. You will need funext.

\end{frame}

\begin{frame}
  \frametitle{Theorem \grey{(impredicative characterization of propositional truncation)}}

\vfill

\grey{(Independently of how it is defined concretely, from the universal property alone.)}

\vfill


\grey{$\underbrace{\dr{\trunc{X}}}_{\text{small type}} \iff \underbrace{\db{\Pi(P:\U), \isProp(P) \to (X \to P) \to P}}_{\text{large type}}$}

\vfill

Moreover, the \db{rhs} is a proposition assuming \db{funext}.

\end{frame}


\begin{frame}
  \frametitle{Univalent logic}

Like Curry-Howard logic, with two differences only:
\begin{itemize}
\vfill \item \mm{\dr{A \vee B} \eqq \db{\trunc{A + B}}}.
\vfill \item \mm{\dr{\exists(x:X), A(x)} \eqq \db{\trunc{\Sigma(x:X),A(x) }}}.
\end{itemize}

\vfill

\grey{This turns out to give exactly the same disjunction and existence as \db{topos logic} and \db{higher order intuitionistic logic}.}

\end{frame}

\begin{frame}
  \frametitle{Example concluded: univalent image}

The image of a function \m{f:X \to Y} is
\M{\dr{\image f} \eqq \db{\Sigma(y:Y), \trunc{\Sigma(x:X), f(x)=y}}.}


\end{frame}


\begin{frame}
  \frametitle{Theorem \grey{(impredicative characterization of univalent logic)}}

We get the usual intuitionistic higher-order logic, which reduces everything to \emph{implication}~\m{\to} and \emph{universal quantification}~\m{\Pi}:

  \begin{itemize}
  \item \vfill \mm{\dr{\bot} \iff \db{\Pi(R:Prop), R}.}
  \item \vfill \mm{\dr{P \wedge Q} \iff \db{\Pi(R:\Prop), (P \to Q \to R) \to R}.}
  \item \vfill \mm{\dr{P \vee Q} \iff \db{\Pi(R:\Prop), (P \to R) \to (Q \to R) \to R}.}
  \item \vfill \mm{\dr{\exists(x:X), P(x)} \iff \db{\Pi(R:\Prop), (\Pi(x:X), P(x) \to R) \to R}.}
  \item \vfill \mm{\dr{\trunc{x=y}} \iff \db{\Pi(P:X \to \Prop), P(x) \to P(y)}}. \grey{(Leibniz principle.)}
  \end{itemize}

\vfill

\grey{(Again, the \dr{lhs} types are small and the \db{rhs} types are large.)}

\end{frame}

\begin{frame}
  \frametitle{Curry--Howard ``excluded middle''}

\dg{Theorem of MLTT+$\trunc{-}$}. The following are logically equivalent:
\begin{enumerate}
\item \m{\Pi(X:\U), X + \neg X}.
\item \m{\Pi(X:\U), \neg \neg X \to X}.
\item \m{\Pi(X:\U), \trunc{X} \to X}.
\item \m{\Pi(X:\U), \Sigma(f:X\to X), \Pi(x,y:X), f(x)=f(y)}.
\end{enumerate}
\begin{itemize}
\item This is more like \db{global choice} than excluded middle. \\[1ex]
\grey{We can pick a point of every non-empty type.}
\item It implies that all types are \db{sets}, making univalent type theory trivial.
\item \dr{False} in the presence of two univalent universes.
\end{itemize}

\end{frame}

\begin{frame}
  \frametitle{Univalent excluded middle}

\vfill

The following are equivalent:
\begin{enumerate}
\vfill \item \m{\Pi(P:\U), \isProp(P) \to P + \neg P}.
\vfill \item \m{\Pi(P:\U), \isProp(P) \to \neg \neg P \to P}.
\end{enumerate}

\vfill

They imply \m{\Pi(X:\U), \trunc{X} \iff \neg\neg X}.

\vfill

\dg{Which is consistent with univalent type theory.}

\end{frame}

\begin{frame}
  \frametitle{\framebox{\dr{Myth:}} propositional truncation erases information}

\vfill

\grey{\dg{It doesn't.} E.g.:}

\vfill

\dg{Theorem of MLTT+$\trunc{-}$}. For any \m{f:\N \to \N},
\M{\trunc{\Sigma(n:\N), f(n)=0} \to \Sigma(n:\N), f(n)=0.}

\vfill

If there is a root of \m{f}, we can find one.

\end{frame}

\begin{frame}
  \frametitle{\dg{Exercise}}

  Prove this.
\end{frame}

\begin{frame}
  \frametitle{Correct formulation of unique existence}

  \begin{itemize}
  \vfill \item
Not \mm{\db{(\Sigma(x:X), A(x))} \times \dr{(\Pi(x,y:X), A(x) \times A(y) \to x=y)}}.
  \vfill \item Instead \m{\isContr(\Sigma(x:X), A(x)).} \\[1ex]

  \grey{Especially when formulating universal properties.}

\vfill \item A unique \m{x:X} such that \m{A(x)} is not enough.

\vfill \item What is really needed is a unique \emph{pair} \m{(x,a)} with \m{x:X} and \m{a:A(x)}. \\[1ex]
\grey{Like in category theory again.} \\[1ex]
\grey{Unless all types are sets.}
  \end{itemize}

\vfill

\end{frame}

\begin{frame}
  \frametitle{Choice just holds in Curry--Howard logic}

Let \m{X,Y:\U} be types and \m{R : X \times Y \to \U} be a relation.
\[
\dr{(\Pi(x:X), \Sigma(y:Y), R(x,y))} \to \db{\Sigma(f:X\to Y), \Pi(x:X), R(x,f(x))}.
\]

\vfill

\grey{Moreover, the implication can be strengthened to a type equivalence.}

\end{frame}

\begin{frame}
  \frametitle{\dg{Exercise}}

  Prove these two claims.
\end{frame}


\begin{frame}
  \frametitle{However, univalent choice implies univalent excluded middle}

\[
\dr{(\Pi(x:X), \trunc{\Sigma(y:Y(x)), R(x,y)})} \to \trunc{\db{\Sigma(f:\Pi(x:X), Y(x)), \Pi(x:X), R(x,f(x))}}.
\]

\vfill

The assumptions are that \m{\isSet{X}} and \m{\isSet{Y(x)}} for all \m{x:X}, and we are given \m{R : (\Sigma(x:X), Y(x)) \to \Prop}.

\vfill

\dg{This form of choice is consistent with univalent type theory.}


\end{frame}

\begin{frame}
  \frametitle{But we do get unique choice}
\end{frame}


\begin{frame}
  \frametitle{Summary of univalent logic}

  \begin{enumerate}
  \vfill \item \m{\Sigma} is used to express given \framebox{structure} or \framebox{data} in general. \\[1ex]
\grey{Cf.\ the type of groups.}

  \vfill \item \db{Truncated} \m{\Sigma} is used to express \framebox{existence}. \grey{(Still constructive.)}
    \begin{itemize}
    \vfill \item But, even better, in practice, one is encouraged to use \m{\Sigma} so that it produces univalent propositions without the need of truncation \grey{(if we can).}
\vfill \item
A crucial example is Voevodsky's primary notion of \db{equivalence}. \\[1ex]
\grey{(But we have seen additional examples.)}
    \end{itemize}
\vfill \item At the moment, it seems to be an art to decide whether  particular mathematical statements should be formulated as giving structure/data or propositions.
\vfill \item But the \db{main point} is that \framebox{univalent mathematics allows the distinction.} \\[1ex]

\grey{(A distinction that is obliterated by the axiom of choice.)}

  \end{enumerate}

\end{frame}

\end{document}
